\documentstyle[12pt]{article}
 
\setlength{\oddsidemargin}{-0.25in}
\setlength{\textwidth}{7.0in}
\setlength{\headheight}{0.0in}
\setlength{\topmargin}{-0.25in}
\setlength{\textheight}{9.0in}

\begin{document}

\pagestyle{empty}

\noindent
{\Large{\bf
  \noindent Math 5620 Fundamentals of Computational Mathematics
  \ \\
  \noindent Fall Semester 2020
}}

\medskip

\noindent {\bf Instructor:} Joe Koebbe

\noindent {\bf Office:} ANSC 209

\noindent {\bf Office Hours:} 10:00-11:50 TTh and 1:30-2:00 TTh or by
appointment.

\noindent {\bf email:} Joe.Koebbe@usu.edu

\noindent {\bf webpage:} http://www.math.usu.edu/~koebbe

\noindent {\bf Office Phone:} 435-797-2825 

\medskip

\noindent {\bf Important Contact Information:} If you call my office phone
number, leave a message on voice mail. Voice mail messages are automatically
forwarded to my email and I can listen and respond at home. The best way to
contact me is directly through email. I read email multiple times each day.

\bigskip

\noindent {\bf USU Course Catalog Description for this Course:}

\medskip

\noindent
{\bf MATH 5620 - Numerical Algorithms for Approximate Solutions of DE 2 credits}
Students solve initial value problems (IVP) and boundary value problems (BVP) in
one dimension using standard methods. Topics include implicit and explicit
methods, local and global error, stability, consistency and convergence,
predictor-corrector methods and Runge-Kutta schemes, multi-step methods, and
finite-difference methods for BVP.  Prerequisite/Restriction: MATH 4610,
MATH 2250, or MATH 2280 with a C- or better
\noindent
This listing includes updates which are effective beginning Fall 2019.

\bigskip

\noindent {\bf Textbook:} \lq\lq Finite Difference Methods for Ordinary and
Partial Differential Equations. \rq\rq\ by Randall J. Leveque

\bigskip

\noindent {\bf General Comments/Polocies on the Course:}  
This course covers the development and implementation of algorithms for the
approximate solution of ordinary and partial differential equations. The course
treats Initial Value Problems (IVPs) and Boundary Value Problems (BVPs)
replacing derivatives of the unknown function with finite difference
approximations. The course will cover the approximate solution of elliptic,
parabolic, and hyperbolic differential equations. The finite difference schemes
created will be analyzed for accuracy, stability and convergence of the
approximations. Methods for handling boundary and initial conditions will be
presented and implemented. 

The course will involve significant computer programming assignments. Students
must know how to write computer code in a high level programming language (e.g;
Python, Fortran, C, C++, Java, and others) or work in a computational platform
like Matlab or Maple. If you choose to use Matlab or Maple, you will be required
to implement your own versions of the algorithms discussed in class. You can use
the intrinsic routines to verify your code. For example, in Matlab there are
simple expressions that will prompt the software within Matlab to compute an
approximate solution of a linear system of equations. The nuts and bolts are
buried deep in the software that Matlab provides. The point of this course is
for each student to be able to fully implement algorithms. Using the software
provided in Matlab does not allow this. There is another reason why students
need to learn all the nuts and bolts that software like Matlab, Mathematica, and
Maple cloak. When you are working in a job and need to ship software to your
clients, Matlab, Mathematica, and Maple require expensive licensing for their
part of the software. These packages are too expensive in most cases. So, you
will need to write your own code. 

Students will be required to obtain a Github account and post homework soltuions
on this account. Students will create a private repository and invite your
instructor to be a \lq\lq collaborator\rq\rq\ so that the assignments can be
turned in electronically. A tutorial on how to do this will be presented in
class.

Parallel algorithms for solving problems have been around for a long time. At
first, these algorithms were interesting, but maintaining parallel codes can be
difficult. In the past decade, the limitations of the physics associated with
computing have put limitations on the speed and capacity of a single CPU.
Computer companies have started to put more processors into their computers with
the idea of improving performance. GPUs have been included to off load the work
needed for displaying graphics and the like onto a specialized parallel
processing card. Ways to take advantage of multiple cores and CPUs and the
GPU onboard your computer have been developed. Students in this course will
gain some knowledge of how to use these tools. In particular, concepts from
the directive based languages, OpenMP and OpenACC, will be presented to show
students the benefits of multicore processing the GPU processing.

\bigskip

\noindent {\bf Grading:} Your grade in the course will be determined by the
following:
\begin{enumerate}
\item Homework will account for a 40\% of a student's grade. Homework must be
      turned in on time. {\bf Late homework will not be accepted under any
      circumstances.} If you must be gone, any homework must be turned in before
      the due date. The only exception to this rule is for a family emergency.
      Also, codes must be carefully documented. An example of a well documented
      code will be discussed in class and a version will be posted on line.
\item Two midterms will be given. Each will cover about one half of the content
      of the course. Each of these midterms will account for 20\% or 40\% of the
      grade earned in the course.
\item A portfolio of routines will be required of each student. The portfolio of
      routines must be written in the form of a software manual. The exact
      form of the portfolio will be discussed in class. Based on past experience
      the formatting will be strict in the sense that the entries in the manual
      will be limited to two pages. The actual format will also be restricted.
      Software manuals that do not meet the formatting requirements will not be
      graded. Part of the reason for putting strict requirements for formatting
      is due to the idea that in a job, you will need to follow the company
      formatting to have your algorithms included in bigger projects.
\end{enumerate}
If students have questions about assignments and midterms, please contact me.
One of the assignments will be to meet with me at least once every two weeks to
discuss progress in the course. This means that 2 times each month during the
semester you will need to show up at office hours or make an appointment to see
me. You will need to do this 8 times during the semester.

\end{document}
